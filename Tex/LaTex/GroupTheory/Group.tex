\documentclass{jsarticle}
\usepackage{amsfonts}
\begin{document}

整数全体の集合$\mathbb{Z}$と
普通の加法の演算子$+$は 以下の法則を満たす。

1. 結合法則
\begin{equation}
    \forall x,y,z \in Z, (x + y) + z = x + (y + z)
\end{equation}

2. 単位元の存在
\begin{equation}
    \forall z, e \in Z, z + e = e + z = z
\end{equation}
この時の、元$e$を単位元とする。

3. 逆元の存在
\begin{equation}
    \forall z,-z \in Z, z + (-z) = (-z) + z = e
\end{equation}
この時の、元$-z$を$z$の逆元とする。

以上の整数全体の集合$\mathbb{Z}$と
普通の加法の演算子$+$の組$(Z,+)$を群と呼ぶ。

\end{document}
